% This example An LaTeX document showing how to use the l3proj class to
% write your report. Use pdflatex and bibtex to process the file, creating
% a PDF file as output (there is no need to use dvips when using pdflatex).

% Modified

\documentclass{l3proj}
\begin{document}
\title{Multi-device Recording System}
\author{Alastair Weir \\
        Gordon Adam \\
        Peter Yordanov \\
        Keir Smith \\
        Georgi Dimitrov}
\date{28 March 2014}
\maketitle
\begin{abstract}

Create a mobile first approach to capturing locational audio centred around
social and political events with the aim of crowdsourcing new insights and
perspectives on events in our cities.

\end{abstract}
\educationalconsent
\tableofcontents
%==============================================================================
\chapter{Introduction}
\label{intro}

%==============================================================================
\chapter{Planning}
\label{Planning}

Plans in here will come in handy

%==============================================================================
\chapter{Research}
\label{Research}

Research in here will come in handy

%==============================================================================
\chapter{Design}
\label{design}

The core of our design sprung from early storyboarding and imagined user
profiles which informed us extensively to what a possible system might look
like. As previously discussed, MDRS was viewed as a tool to democratise access
to information at large scale events and share intimate moments or memories of a
place and time through an interactive storytelling medium.


%==============================================================================
\chapter{Implementation}
\label{impl}

In this chapter, we describe how the implemented the system.

%------------------------------------------------------------------------------
\section{Web Application}

Blah blah blah
Blah blah blah
Blah blah blah
Blah blah blah

% - - - - - - - - - - - - - - - - - - - - - - - - - - - - - - - - - - - - - - -
\subsection{Server and Django configuration}

Blah blah blah
Blah blah blah
Blah blah blah
Blah blah blah

%------------------------------------------------------------------------------
\section{Android Application}

\begin{enumerate}
\item Blah blah blah
\item Blah blah blah
\item Blah blah blah
\item Blah blah blah
\end{enumerate}



%==============================================================================
\chapter{Evaluation}

We evaluated the project by...

%==============================================================================
\chapter{Challenges}
\label{Challenges}

A number of the challenges we encountered were through a lack of previous
experience and inaccurate expectations held by each member of the team. The
development process offered a lot of flexibility in finding weaknesses in our
ideas and restructuring them forge better working practices and develop new
insights into the workings of a small scale agile team.

\subsection{Technological}

Our main technological issues were over-complication
and over dependence on multiple services for specific channels of communication.
As previously described we embarked on the project with the intention of using a
Redmine instance to keep track of project details, GitHub as our VCS and
Facebook for communication. This convoluted mix made it near impossible to keep
track of discussion about specific issues and problems. While the messaging
component of Facebook was a thoroughly successful choice for social
communication about our Team's work, the group page wasn't as successful. Some
features such as a tracker of who has seen a particular message, it failed due
to posts not being kept in chronological order and no way to easily track
important information.

Through other project work and happenstance, the team discovered GitHub's issue
tracker and wiki features which are individual to each repository. These quickly
became the default means of tracking progress for MDRS, replacing Redmine. The
customisable labels and ability to cross reference issues from commits were
invaluable. GitHub's fast and responsive web interface scaled well across
devices and meant everyone was able to be involved in decisions and contribute
issues to work on. Most of all it succeeded due to the team members being on the
service already whereas Redmine required an intent to go visit it.

Surprisingly email notifications became a great source of information. As
default, GitHub sends out emails for every comment or new issue created. While
filling up inboxes, this device agnostic communication platform made for great
commute reading. While notifications can easily be lost in a endless-scroll, the
emails were small, actionable pieces of key information to keep track of the
project's direction as a whole.

In future, due to the distributed nature of the team and constantly shifting
focus of attention for different deadlines the team would leverage email more.
Weekly status reports would serve as talking points to hopefully make meetings
more productive, giving members time to prepare their thoughts. These would also
serve as evidence of communication and would be easily accessible at a later
date in a chronological order.

Other technological issues revolved around a lack of experience and knowledge at
the beginning of the project. Issues such as handling dependencies from pip with
a requirements.txt poorly made setting up a new virtual environment a challenge.
This was resolved later but mean the team lacked consistency of tools being used
early on.

A major misunderstanding came with the team using VCS. To begin with we did not
understand the workflow of working and testing locally then pushing to the
repository. Initially we were all using SSH to access our server and then
working directly on the server, creating conflicts and locking file issues. This
was quickly resolved when the team researched the issue, quickly changing to the
correct workflow.

\subsection{Organisational}
Our main issue was in poor communication. As
previously discussed we struggled with too many communication channels but this
improved with more face to face interactions in the laboratory and using the
social chat group more for continuous, incremental updates and communication in
the team.

At the beginning of the development cycle, agile roles were assigned for scrum
master, communications with supervisor, librarian and developer. Throughout
development these roles merged and adhered to the agile principles even further.
Due to the ad-hoc nature of the development team and shifting focus amongst
projects, a different team member shifted into the scrum master role who handled
the majority of secretarial jobs, keeping the team on track with tasks assigned
to them to complete. These tasks were assigned taking into consideration each
team member’s strengths and interests. This increased productivity within the
team, pushing our development further and allowing us to expand on the feature
set included in the final project.

%==============================================================================
\chapter{Future Work}
\label{Future Work}

Future work in here will come in handy

%==============================================================================
\chapter{Conclusion}

A great project!

%==============================================================================
\section{Contributions}

Conclusion here

%==============================================================================
\bibliographystyle{plain}
\bibliography{example}
\end{document}
